
% -------------------------------------------------------------
% Paquetes
% -------------------------------------------------------------
\usepackage{microtype}
\usepackage{commath}
% Configura el espacio tipográfico por razones estéticas
\usepackage{cmbright}                                 % Tipografía (Computer Modern Bright)
\usepackage[colorlinks = true,
    allcolors = black]{hyperref}              % Hiper-referencias en el PDF
\usepackage{subfigure}    % Entorno de multiples figuras
\usepackage{amsmath}   % Matemáticas estándar
\DeclareMathOperator{\arcsec}{arcsec}
\DeclareMathOperator{\arccot}{arccot}
\DeclareMathOperator{\arccsc}{arccsc}
\usepackage{amssymb}                                  % Símbolos matemáticos estándar
\usepackage{graphicx}                                 % Para incluir imágenes
\usepackage{booktabs}                                 % Tablas profesionales
\usepackage[spanish]{babel}                           % Definición del lenguaje
\usepackage[a4paper,                                  % Tamaño del papel
    top = 25mm,                               % Margen superior
    bottom = 25mm,                            % Margen inferior
    left = 25mm,                              % Margen izquierdo
    right = 25mm]{geometry}                   % Margen derecho
\usepackage[dvipsnames]{xcolor}                       % Colores avanzados
\usepackage{soul}                                     % Resaltar texto y otras cosas
\usepackage{caption}                                  % Descripciones en objetos flotantes
\usepackage{textcomp}                                 % Símbolo de euro (obsoletar?)
\usepackage{ifthen}                                   % Condicionales if/then
\usepackage{fmtcount}                                 % Convertir enteros a palabras (obsoletar?)
\usepackage[per-mode = symbol]{siunitx}	              % Unidades del S.I.U
\usepackage[version=4]{mhchem}                        % Símbolos químicos
\usepackage{totcount}                                 % Mostrar contador de puntaje al princpio
\usepackage{xstring}                                  % Para manipular strings
\usepackage{multicol}                                 % Entorno de multiples columnas (FS)
\usepackage{lipsum}                                   % Texto Lorem Ipsum (obsoletar?)

\usepackage{fontspec}
\newfontfamily{\emoji}{emoji.ttf}[Renderer=Harfbuzz]

\usepackage{wasysym}                                  % Símbolos para preguntas de alternativas

% -----------------------------------------------------
% Definiciones del Documento
% -----------------------------------------------------
\pointsinrightmargin                                  % Puntos en el margen derecho
\pointformat{\bfseries(\themarginpoints)}  % PUNTOS, agregar condicional!
% Puntos en negrita y formato de paréntesis
\pointformat{}
\renewcommand{\partlabel}{\bfseries{(\thepartno)}}    % Números de parte negrita
\renewcommand{\subpartlabel}{\bfseries{\thesubpart)}} % Números de parte negrita
\pagestyle{headandfoot}                               % Header y Footer de la solución
\header{}{}{}                                         % Header
\footer{}{}{\theCourseCode{}                          % Footer
    \theCourse{} --- \theExamDate{}
    (Estudiatón)}
\graphicspath{{imagenes/}}                              % Directorio con las imágenes
\definecolor{myRed}{RGB}{184,0,1}                     % Nuevo gris
\definecolor{myGrey}{RGB}{211,211,211}                % Nuevo rojo
\checkedchar{\large $\CIRCLE$}
\setlength{\fboxrule}{0pt}

% -----------------------------------------------------
% Versiones
% -----------------------------------------------------
\newcommand{\theVersion}{1.3}
\newcommand{\theVersionDate}{24-May-2020}

% -----------------------------------------------------
% Nuevos Comandos
% -----------------------------------------------------
\newtotcounter{pointsCounter}                         % Contador del número de puntos por parte
\newtotcounter{numberOfSections}                      % Contador de número de secciones
\newtotcounter{numberOfQuestions}                     % Contador de número de preguntas

% -----------------------------------------------------
% Estructura del Documento - Preguntas y secciones
% -----------------------------------------------------
\ifthenelse{\equal{\hasSections}{Yes} \OR \equal{\hasSections}{Y} \OR \equal{\hasSections}{yes} \OR \equal{\hasSections}{y} \OR \equal{\hasSections}{YES}}
% Si es que tiene secciones
{\newcommand{\sectionStart}{                           % Initiate new sections
        \setcounter{question}{0}                          % (Re)initiate questions counter
        \setcounter{figure}{0}                            % Initiate figure counter
        \setcounter{table}{0}                             % Initiate table counter
        \qformat{\ifthenelse{\equal{\thequestion}{1}}
            {\fbox{\parbox{16cm}{\textbf{\large SECCIÓN \Alph{numberOfSections} \hfill\vrule depth 0.5em width 0pt \\
                            Question \Alph{numberOfSections}\thequestion}} \hfill\vrule depth 1.5em width 0pt}}
            {\fbox{\parbox{16cm}{\large \textbf{Pregunta \Alph{numberOfSections}\thequestion}} \hfill\vrule depth 0.8em width 0pt}}
        }
        \stepcounter{numberOfSections}                    % Incrementar contador
    }}
% Si no hay secciones
{   \setcounter{question}{0}                          % Reinicializar contador de preguntas
    \setcounter{figure}{0}                            % Iniciar contador de figuras
    \setcounter{table}{0}                             % Iniciar contador de tablas
    \qformat{
        {\fbox{\parbox{16cm}{\large \textbf{Pregunta \thequestion}} \hfill\vrule depth 0.8em width 0pt}}
    }
}

\newcommand{\callQuestion}[2]{                        % Nueva pregunta
    \cleardoublepage \input{#1}                       % Iniciar pregunta en nueva página
    \addtocounter{pointsCounter}{#2}                  % Incrementar puntos
    \addtocounter{numberOfQuestions}{1}               % Incrementar preguntas
    \stepcounter{figure}                              % Incrementar figuras
    \stepcounter{table}                               % Incrementar tablas
}

% Define la hoja de fórmulas
\newcommand{\callFormulaSheet}[1]{
    \ifthenelse{\equal{\formulaSheet}{Yes} \OR \equal{\formulaSheet}{Y} \OR \equal{\formulaSheet}{yes} \OR \equal{\formulaSheet}{y} \OR \equal{\formulaSheet}{YES}}{\cleardoublepage \input{#1}}{}
}

\newcommand{\headingFormulaSheet}{
    \pagestyle{empty}
    \setlength{\textwidth}{180mm}
    \setlength{\textheight}{265mm}
    \setlength{\topmargin}{-20mm}
    \setlength\oddsidemargin{-12mm}
    \setlength\evensidemargin{-12mm}
    \setlength{\columnsep}{10mm}
    \setlength{\columnseprule}{0.4pt}
    \setlength{\jot}{5pt}
    \setlength\parindent{0pt}
    %
    \twocolumn
    \fcolorbox{white}{myGrey}{\begin{minipage}{20em}
            \centering \vspace{4mm} {\Large \textbf{Hoja de Fórmulas}} \\
            \theCourse\ (\theCourseCode) \\
            {\small \theExamDate{} (\theDiet)} \vspace{3mm}
        \end{minipage}}
}

% -----------------------------------------------------
% Títulos y formatos
% -----------------------------------------------------

%\qformat{\textbf{\large Question \thequestion} \hfill\vrule depth 1.5em width 0pt}

% Formato del título de la pregunta - solo las numeradas y no en partes
\pointname{}                                          % No title for marks
\pointsdroppedatright                                 % Puntaje por pregunta
\framedsolutions                                      % Marco para soluciones
\addto\captionsenglish{                               % Formato de etiquetas de figura
    %    \renewcommand\figurename{Figure \thequestion .}}
    \renewcommand\figurename{Figura}}
\addto\captionsenglish{                               % Etiquetas de tabla
    %    \renewcommand\tablename{Table \thequestion .}}
    \renewcommand\tablename{Tabla}}
\makeatletter
\def\fnum@figure{\figurename~\Alph{numberOfSections}\thefigure}
\def\fnum@table{\tablename~\Alph{numberOfSections}\thetable}
\makeatother
\renewcommand{\solutiontitle}{
    \noindent\textbf{Solución:}\par\noindent}  % Renombrar soluciones
