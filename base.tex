% !TeX root = ./documento.tex
% -------------------------------------------------------------
% Paquetes
% -------------------------------------------------------------
\usepackage{microtype}
\usepackage{commath}                                  % Configura el espacio tipográfico por razones estéticas
\usepackage{cmbright}                                 % Tipografía (Computer Modern Bright)
\usepackage[colorlinks = true,
    allcolors = black]{hyperref}                      % Vínculos
\usepackage{subfigure}                                % Entorno de multiples figuras
\usepackage{amsmath}                                  % Matemáticas estándar
\usepackage{amssymb}                                  % Símbolos matemáticos estándar
\usepackage{graphicx}                                 % Para incluir imágenes
\usepackage{booktabs}                                 % Tablas profesionales
\usepackage[spanish, es-noshorthands]{babel}          % Definición del lenguaje

% Deshabilitado: Tamaño A4
% \usepackage[a4paper,                                % Tamaño del papel
%     top = 25mm,                                     % Margen superior
%     bottom = 25mm,                                  % Margen inferior
%     left = 25mm,                                    % Margen izquierdo
%     right = 25mm]{geometry}                         % Margen derecho

\usepackage[dvipsnames]{xcolor}                       % Colores avanzados
\usepackage{soul}                                     % Resaltar texto y otras cosas
\usepackage{caption}                                  % Descripciones en objetos flotantes
\usepackage{ifthen}                                   % Condicionales if/then
\usepackage{fmtcount}                                 % Convertir enteros a palabras
\usepackage[per-mode = symbol]{siunitx}	              % Unidades del S.I.U
\usepackage[version=4]{mhchem}                        % Símbolos químicos
\usepackage{totcount}                                 % Mostrar contador de puntaje al princpio
\usepackage{xstring}                                  % Para manipular strings
\usepackage{multicol}                                 % Entorno de multiples columnas (FS)
\usepackage{mfirstuc}                                 % Agrega comandos para mayúsculas
\usepackage[no-math]{fontspec}                        % Manejo tipográfico manual
\usepackage{wasysym}                                  % Símbolos para preguntas de alternativas
\usepackage{dirtytalk}                                % Citas
\usepackage{etoolbox}                                 % if/else modernos
% -----------------------------------------------------
% Definiciones del Documento
% -----------------------------------------------------

\renewcommand{\partlabel}{\bfseries{(\thepartno)}}     % Números de parte negrita
\renewcommand{\subpartlabel}{\bfseries{\thesubpart{}}} % Números de parte negrita
\pagestyle{headandfoot}                                % Header y Footer de la solución
\header{}{}{}                                          % Header
\footer{}{}{\codigoDelCurso{}                          % Footer
    \elCurso{} --- \fecha{}
    (\nombreDocumento)}
\graphicspath{{imagenes/}}                             % Directorio con las imágenes
\definecolor{myRed}{RGB}{184,0,1}                      % Nuevo gris
\definecolor{myGrey}{RGB}{211,211,211}                 % Nuevo rojo
\checkedchar{\large \(\CIRCLE\)}
\setlength{\fboxrule}{0pt}

% -----------------------------------------------------
% Versiones
% -----------------------------------------------------
\newcommand{\theVersion}{1.3}
\newcommand{\theVersionDate}{24-May-2020}

% -----------------------------------------------------
% Nuevos Comandos
% -----------------------------------------------------
\newtotcounter{contadorDePuntos}                      % Contador del número de puntos por parte
\newtotcounter{numeroDeSecciones}                     % Contador de número de secciones
\newtotcounter{numeroDePreguntas}                     % Contador de número de preguntas
\DeclareMathOperator{\arcsec}{arcsec}                 % Funciones Trigonométricas Inversas
\DeclareMathOperator{\arccot}{arccot}
\DeclareMathOperator{\arccsc}{arccsc}

% -----------------------------------------------------
% Lógica Interna
% -----------------------------------------------------

% Thanks Phelype Oleinik for his awesome answer on StackExchange suggesting this. https://tex.stackexchange.com/users/134574/phelype-oleinik
\newrobustcmd\sanitizadorSiNo[1]{%
    \MakeLowercase{\gdef\noexpand\internalCondition{#1}}%
    \ifboolexpr{%
        test {\expandafter\ifstrequal\expandafter{\internalCondition}{yes}}
        or
        test {\expandafter\ifstrequal\expandafter{\internalCondition}{yas}}
        or
        test {\expandafter\ifstrequal\expandafter{\internalCondition}{si}}
        or
        test {\expandafter\ifstrequal\expandafter{\internalCondition}{sí}}
        or
        test {\expandafter\ifstrequal\expandafter{\internalCondition}{s\IeC{\'\i}}}
        or
        test {\expandafter\ifstrequal\expandafter{\internalCondition}{s\IeC{\'i}}} % For SÍ
    }}

\newrobustcmd\sanitizadorPorResponder[1]{%
    \MakeLowercase{\gdef\noexpand\internalCondition{#1}}%
    \ifboolexpr{%
        test {\expandafter\ifstrequal\expandafter{\internalCondition}{todas}}
        or
        test {\expandafter\ifstrequal\expandafter{\internalCondition}{todo}}
        or
        test {\expandafter\ifstrequal\expandafter{\internalCondition}{toda}}
        or
        test {\expandafter\ifstrequal\expandafter{\internalCondition}{all}}
    }}
\newrobustcmd\sanitizadorLibroAC[1]{%
    \MakeLowercase{\gdef\noexpand\internalCondition{#1}}%
    \ifboolexpr{%
        test {\expandafter\ifstrequal\expandafter{\internalCondition}{abierto}}
        or
        test {\expandafter\ifstrequal\expandafter{\internalCondition}{apuntes}}
    }}
\newrobustcmd\sanitizadorEsEvaluacion[1]{%
    \MakeLowercase{\gdef\noexpand\internalCondition{#1}}%
    \ifboolexpr{%
        test {\expandafter\ifstrequal\expandafter{\internalCondition}{interrogación}}
        or
        test {\expandafter\ifstrequal\expandafter{\internalCondition}{interrogacion}}
        or
        test {\expandafter\ifstrequal\expandafter{\internalCondition}{control}}
        or
        test {\expandafter\ifstrequal\expandafter{\internalCondition}{evaluación}}
        or
        test {\expandafter\ifstrequal\expandafter{\internalCondition}{evaluacion}}
    }}

\newrobustcmd\siHojaDeFormulas{%
    \expandafter\sanitizadorSiNo
    \expandafter{\hojaDeFormulas}}%
\newrobustcmd\siTieneSecciones{%
    \expandafter\sanitizadorSiNo
    \expandafter{\tieneSecciones}}%
\newrobustcmd\siConRespuestas{%
    \expandafter\sanitizadorSiNo
    \expandafter{\conRespuestas}}%
\newrobustcmd\siSoportaEmojis{%
    \expandafter\sanitizadorSiNo
    \expandafter{\soportaEmojis}}%
\newrobustcmd\siMostrarPuntos{%
    \expandafter\sanitizadorSiNo
    \expandafter{\mostrarPuntos}}%
\newrobustcmd\siConCalculadoras{%
    \expandafter\sanitizadorSiNo
    \expandafter{\conCalculadoras}}%
\newrobustcmd\siRespuestaPorSeparado{%
    \expandafter\sanitizadorSiNo
    \expandafter{\respuestaPorSeparado}}%
\newrobustcmd\siResponderTodo{%
    \expandafter\sanitizadorPorResponder
    \expandafter{\preguntasPorResponder}}%
\newrobustcmd\siEsEvaluacion{%
    \expandafter\sanitizadorEsEvaluacion
    \expandafter{\tipoDocumento}}%
\newrobustcmd\siEsLibroAbierto{%
    \expandafter\sanitizadorLibroAC
    \expandafter{\libroAC}}%
\ExplSyntaxOff

\siEsEvaluacion{\newcommand{\familiaDocumento}{evaluación}}{\newcommand{\familiaDocumento}{guía}}
% Maneja el soporte de emojis
\siSoportaEmojis{
    \newfontfamily{\emoji}{emoji.ttf}[Renderer=Harfbuzz]}  %  if "Enunciados con emojis" - ΠCAi
{} % else

% -----------------------------------------------------
% Estructura del Documento - Preguntas y secciones
% -----------------------------------------------------
\siTieneSecciones
{\newcommand{\sectionStart}{                              % Iniciar sección
        \setcounter{question}{0}                          % Reiniciar contador preguntas
        \setcounter{figure}{0}                            % Iniciar contador de figuras
        \setcounter{table}{0}                             % Iniciar contador de tablas
        \qformat{\ifthenelse{\equal{\thequestion}{1}}
            {\fbox{\parbox{16cm}{\textbf{\large SECCIÓN \Alph{numeroDeSecciones} \hfill\vrule depth 0.5em width 0pt \\
                            Question \Alph{numeroDeSecciones}\thequestion}} \hfill\vrule depth 1.5em width 0pt}}
            {\fbox{\parbox{16cm}{\large \textbf{Pregunta \Alph{numeroDeSecciones}\thequestion}} \hfill\vrule depth 0.8em width 0pt}}
        }
        \stepcounter{numeroDeSecciones}               % Incrementar contador
    }}
% Si no hay secciones
{   \setcounter{question}{0}                          % Reinicializar contador de preguntas
    \setcounter{figure}{0}                            % Iniciar contador de figuras
    \setcounter{table}{0}                             % Iniciar contador de tablas
    \qformat{
        {\fbox{\parbox{16cm}{\large \textbf{Pregunta \thequestion}} \hfill\vrule depth 0.8em width 0pt}}
    }
}
% Invocador misterioso de preguntas
\newcommand{\traerPregunta}[2]{                       % Nueva pregunta
    \cleardoublepage \input{#1}                       % Iniciar pregunta en nueva página
    \addtocounter{contadorDePuntos}{#2}               % Incrementar puntos
    \addtocounter{numeroDePreguntas}{1}               % Incrementar preguntas
    \stepcounter{figure}                              % Incrementar figuras
    \stepcounter{table}                               % Incrementar tablas
}

% Define la hoja de fórmulas
\newcommand{\traerHojaDeFormulas}[1]{
    \ifthenelse{
        \equal{\hojaDeFormulas}{Yes} \OR{} \equal{\hojaDeFormulas}{Y} \OR{} \equal{\hojaDeFormulas}{yes} \OR{} \equal{\hojaDeFormulas}{y} \OR{} \equal{\hojaDeFormulas}{YES} \OR{} \equal{\hojaDeFormulas}{Si} \OR{} \equal{\hojaDeFormulas}{Sí} \OR{} \equal{\hojaDeFormulas}{Verdadero} \OR{} \equal{\hojaDeFormulas}{si} \OR{} \equal{\hojaDeFormulas}{sí} \OR{} \equal{\hojaDeFormulas}{verdadero} \OR{} \equal{\hojaDeFormulas}{SÍ} \OR{} \equal{\hojaDeFormulas}{SI} \OR{} \equal{\hojaDeFormulas}{VERDADERO}
    }
    {\cleardoublepage \input{#1}}{}
}
% Cabecera de la hoja de fórmulas
\newcommand{\headingFormulaSheet}{
    \pagestyle{empty}
    \setlength{\textwidth}{180mm}
    \setlength{\textheight}{265mm}
    \setlength{\topmargin}{-20mm}
    \setlength\oddsidemargin{-12mm}
    \setlength\evensidemargin{-12mm}
    \setlength{\columnsep}{10mm}
    \setlength{\columnseprule}{0.4pt}
    \setlength{\jot}{5pt}
    \setlength\parindent{0pt}
    %
    \twocolumn
    \fcolorbox{white}{myGrey}{\begin{minipage}{20em}
            \centering \vspace{4mm} {\Large \textbf{Hoja de Fórmulas}} \\
            \elCurso\ (\codigoDelCurso) \\
            {\small \fecha{} (\tipoDocumento)} \vspace{3mm}
        \end{minipage}}
}

% -----------------------------------------------------
% Títulos y formatos
% -----------------------------------------------------

%\qformat{\textbf{\large Question \thequestion} \hfill\vrule depth 1.5em width 0pt}

\pointsinrightmargin{}                                % Puntos en el margen derecho
\siMostrarPuntos{\pointformat{\bfseries(\themarginpoints)}}{\pointformat{}}

% Formato del título de la pregunta - solo las numeradas y no en partes
\pointname{}                                          % No title for marks
\pointsdroppedatright{}                               % Puntaje por pregunta
\framedsolutions{}                                    % Marco para soluciones
\addto\captionsenglish{                               % Formato de etiquetas de figura
    %    \renewcommand\figurename{Figure \thequestion .}}
    \renewcommand\figurename{Figura}}
\addto\captionsenglish{                               % Etiquetas de tabla
    %    \renewcommand\tablename{Table \thequestion .}}
    \renewcommand\tablename{Tabla}}
\makeatletter
\def\fnum@figure{\figurename~\Alph{numeroDeSecciones}\thefigure}
\def\fnum@table{\tablename~\Alph{numeroDeSecciones}\thetable}
\makeatother
\renewcommand{\solutiontitle}{
    \noindent\textbf{Solución:}\par\noindent}  % Renombrar soluciones