% Definiciones de comandos, para reutilizar secuencias frecuentes o ahorrar código

\newcommand{\mytitle}{Ayudantía 17}
\newcommand{\tema}{Capacitancia y Materiales Dieléctricos}
\newcommand{\fechaentrega}{11-10-2022}
\newcommand{\ayudante}{Iván Vergara Lam}
\newcommand{\mailuc}{ivl@uc.cl}

\newcommand{\siglacurso}{FIS1533}
\newcommand{\nombrecurso}{Electricidad y Magnetismo}
\newcommand{\profesor}{Edgardo Stockmeyer}
\newcommand{\numseccion}{3}
\newcommand{\semestre}{Segundo Semestre del 2022}

\pagestyle{fancy}
\fancyhf{}
\renewcommand{\headrulewidth}{0pt}
\renewcommand{\footrulewidth}{0.35pt}

% Ubicación de figuras
\graphicspath{{./figuras/}}

% Definir color de hipervínculos
\hypersetup{
  colorlinks=false,
  linkbordercolor=0.96 0.60 0.14,
  urlbordercolor=0.96 0.60 0.14
}

\lfoot{\fechaentrega \hfill \siglacurso \ -- \mytitle \hfill Página \thepage{} de \pageref{LastPage}}