%\title{Ayudantía 9-MAT1620}


\documentclass[12pt,spanish]{article}
\usepackage[spanish]{babel}
\usepackage{latexsym}
\usepackage{amssymb}
\usepackage{amsfonts}
\usepackage{amsmath}
\usepackage{graphicx}
\usepackage[latin1]{inputenc}


%\setlength{\textwidth}{5.9in} \textwidth 5.9in
%\setlength{\textheight}{9.8in} \textheight 9.8in \oddsidemargin1.3cm
%\setlength{\topmargin}{-2 true cm} \setlength{\parskip}{10pt}

\setlength{\textwidth}{16cm}
%\textwidth 14cm
\setlength{\textheight}{30.5cm}
%\textheight 21cm
\setlength{\topmargin}{-2.00 true cm}
\setlength{\oddsidemargin}{0.20cm}
\setlength{\evensidemargin}{0.55cm} \setlength{\parskip}{10pt}
\newlength{\mpag}
\setlength{\mpag}{7cm}
\pagestyle{empty}

\pagestyle{empty}


\DeclareMathOperator{\Ker}{{Ker}} \DeclareMathOperator{\Ima}{{Im}}
\newcommand{\R}{\mathbb R}
\newcommand{\N}{\mathbb N}
\newcommand {\pts}[1] {{\bf[#1 pts.]}}


\begin{document}

\noindent PONTIFICIA UNIVERSIDAD CAT\' OLICA DE CHILE\\
Facultad de Matem\' atica\\
Segundo Semestre 2014\\


\noindent{\LARGE{\centerline {\bfseries {Ayudant\'ia 9 }}}}
\noindent {\centerline {\bfseries{ Series de Potencia }}
\noindent{\centerline{MAT1620 - C\' alculo II (Secci\'on 6)}}\\
\noindent{\centerline{Cristian Vald\' es Pimentel (\em cuvaldes@uc.cl)}}\\
\begin{enumerate}
\item Acudiendo al resultado de la serie geom\'etrica, demostrar para que aquellos valores de $x$ tales que $|x| < 1 $, se tiene que $\displaystyle{ \sum_{k=0}^\infty \frac{(k+1)(k+2)}{2!} x^k = \frac{1}{(1-x)^3} }$
\item Determine el intervalo de convergencia de
$$\displaystyle{\sum_{n=1}^\infty \frac{e^{-2/n}x^n}{(n+1)2^n}}$$
\item Sea la siguiente funci\'on definida mediante una serie de potencia.
$$\displaystyle{J(x)= \sum_{k=0}^\infty \frac{(-1)^k}{k!(k+1)!} \Bigg( \frac{x}{2}  \Bigg)^{2k+1}  }$$
Demuestre que el radio de convergencia son todos los reales.
\item Utilizando series de potencia , calcule $\displaystyle{\sum_{n=1}^{\infty} \frac{(-1)^{n+1}}{n}}$

\end{enumerate}
\end{document}
