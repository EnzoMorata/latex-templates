%%%%%%%%%%%%%%%%%%%%%%%%%%%%%%%%%%%%%%%%%%%%%%%%%%%%%%%%%%%%%%%
% IMPORTANTE: Cambiar el compilador a XeLaTeX en las opciones %
%     Si se quiere hacer "doble enter" usar: \vspace{2ex}     %
%               Creditos Github: @diegocostares               %
%     ADVERTENCIA: Al compilar la plantilla por defecto se    %
%     solapa el contenido, se soluciona una vez se escribe    %
%%%%%%%%%%%%%%%%%%%%%%%%%%%%%%%%%%%%%%%%%%%%%%%%%%%%%%%%%%%%%%%
\documentclass{style}

\begin{document}
%%%%%%%%%%%%%%%%% RENOMBRE %%%%%%%%%%%%%%%%%
\graphicspath{ {./img/} }
\renewcommand\cftsecfont{\normalsize}
\renewcommand\cftsecpagefont{\normalsize}
\renewcommand{\contentsname}{\normalsize Tabla de Contenido}
\renewcommand{\listfigurename}{\normalsize Índice de Ilustraciones}
\renewcommand{\listtablename}{\normalsize Índice de Tablas}
\renewcommand{\tablename}{Tabla}
\renewcommand{\figurename}{Figura}
\renewcommand*{\lstlistingname}{Código}

%%%%%%%%%%%%%%%%% PORTADA %%%%%%%%%%%%%%%%%
\ThisCenterWallPaper{1}{content/portada.pdf}
\portada{Nombre completo} % Insertar nombre acá
        {Correo electronico} % Insertar correo acá
        {Empresa}
        {Diploma} % Insertar empresa acá
        {\today} % Insertar fecha

%%%%%%%%%%%%%%%%% ABSTRACT %%%%%%%%%%%%%%%%%
% Resumen ejecutivo en inglés (Máximo una página o entre 200 a 400 palabras)
% - Contexto general
% - Objetivos o interés de la empresa a ser logrado al finalizar la práctica
% - actividades realizadas
% - Metodologías
% - Principales Resultados
% - Conclusiones generales o del informe tecnico

\selectlanguage{British}

\noindent{\Large\textbf{Abstract}\par\vspace{1ex}}

To fill out the executive summary in English (200-400 words), follow these steps. Start with a brief overview of the context. State the company's goals or objectives for the project's conclusion. List the tasks carried out during the project. Explain the methodologies used. Highlight the main outcomes. Conclude with general findings or conclusions from the technical report.

\selectlanguage{Spanish}
\newpage

%%%%%%%%%%%% Numeración paginas %%%%%%%%%%%
\pagenumbering{arabic}
%%%%%%%%%%%%%%%%% INDICES %%%%%%%%%%%%%%%%%
\tableofcontents \newpage
\listoftables \newpage
\listoffigures \newpage
%%%%%%%%%%%%%%%%% ESPACIADO %%%%%%%%%%%%%%%%%
\setstretch{1.15}
%%%%%%%%%%%%%%%%% CONTENIDO %%%%%%%%%%%%%%%%%

% Informe técnico (Máximo 20 páginas)
\section{Reporte técnico}

% =============================================
\subsection{Identificación y Contextualización}
% =============================================

\subsubsection{Contextualización del Lugar de Práctica}
    - Características de la empresa y su área
    - Tamaño de la empresa en relación al rubro

\subsubsection{Descripción del Proyecto y Actividades Realizadas}
    - Objetivo general y descripción técnica
    - Plazos y alcance del proyecto
    - Involucrados en el desarrollo de las actividades

\subsubsection{Objetivo e Interés Específico de la Empresa}
    - Identificación del objetivo específico a lograr
    - Planteamiento de problema o pregunta a resolver

\subsubsection{Antecedentes y Causas del Problema}
    - Detalle de los antecedentes y causas del problema
    - Indicación de la importancia de resolverlo

% =============================================
\subsection{Generación y Justificación}
% =============================================

\subsubsection{Metodología y Herramientas Propuestas}
    - Propone una metodología, herramientas y/o modelos para el análisis, diseño, desarrollo y solución del objetivo/problema (utiliza supuestos si no fueron aplicados durante el periodo de práctica).
    - Utiliza recursos visuales (e.g: como diagramas de flujo, figuras, gráficos, tablas, etc.) que facilitan el entendimiento de las labores realizadas o metodologías utilizadas.

\subsubsection{Análisis, Herramientas y Software para la Resolución del Problema}
    - Describe los análisis, mediciones o aplicaciones y los softwars necesarios para la resolución del problema/demanda (utiliza supuestos si no fueron aplicados durante el periodo de práctica).
    - Si no se hizo antes, utilizar recursos visuales


\subsubsection{Justificación de decisiones metodológicas y de recursos}
    - Justifica las decisiones tomadas en cuanto a la metodología o los recursos utilizados de acuerdo a las características del problema/objetivo abordado y del lugar de realización de la práctica.
    - Si no se hizo antes, utilizar recursos visuales

% =============================================
\subsection{Evaluación y Conclusiones}
% =============================================

\subsubsection{Resultados y Análisis}
    - Describe los resultados obtenidos utilizando indicadores concretos o mediciones reales o estimadas.

\subsubsection{Conclusiones sobre los Resultados}
    - Concluye acerca de los resultados obtenidos, considerando limitaciones y relevancia de los análisis realizados.
    - Discute las implicancias futuras de las decisiones tomadas y plantea posibles modificaciones y pasos a seguir.

\subsubsection{Evaluación de Cumplimientos de Objetivos de la Empresa}
    - Concluye acerca del cumplimiento o incumplimiento de los objetivos o intereses de la empresa.

\subsubsection{Impacto y Relevancia de los Resultados}
    - Justifica adecuadamente el impacto o relevancia de los resultados obtenidos, considerando aspectos como eficiencia, productividad, clima laboral, etc.



%%%%%%%%%%%%%%%%% BIBLIOGRAFIA %%%%%%%%%%%%%%%%%
\newpage
% Hay 2 formas de agregar bibliografia:
% 1. Agregar una bibliografia en un archivo .bib (Es super automatico)
% \bibliographystyle{apacite}
% \bibliography{mybib.bib} % Requiere crear un archivo .bib

% 2. Agregar una bibliografia en un archivo .tex
% (Es manual, pero comodo para los que no conoce el bibtex)
\indent
\section{Referencias bibliográficas}
\leftskip 0.3in
\parindent -0.3in

%%%%%%%%%%%%%%% ESCRIBIR DEBAJO EN FORMATO APA %%%%%%%%%%%%%%%
% las lineas de arriba permiten que tenga una sangría francesa.

%%%%%%%%%%%%%%%%% ANEXOS %%%%%%%%%%%%%%%%%

\begin{center}
\vspace*{\fill}
    
\textbf{ANEXOS}
\addcontentsline{toc}{section}{ANEXOS}
\vspace*{\fill}
\end{center}


%%%% para agregar sectiones o subsecciones al anexo ocupar 
%%%% \addcontentsline{toc}{section}{--title name--}
%%%% \textbf{--title name--}
\newpage
\addcontentsline{toc}{section}{ANEXO A: Diagrama de flujo...}
\textbf{ANEXO A: Diagrama de flujo...}


\end{document}
