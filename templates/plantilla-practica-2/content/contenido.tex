% Informe técnico (Máximo 20 páginas)
\section{Reporte técnico}

% =============================================
\subsection{Identificación y Contextualización}
% =============================================

\subsubsection{Contextualización del Lugar de Práctica}
    - Características de la empresa y su área
    - Tamaño de la empresa en relación al rubro

\subsubsection{Descripción del Proyecto y Actividades Realizadas}
    - Objetivo general y descripción técnica
    - Plazos y alcance del proyecto
    - Involucrados en el desarrollo de las actividades

\subsubsection{Objetivo e Interés Específico de la Empresa}
    - Identificación del objetivo específico a lograr
    - Planteamiento de problema o pregunta a resolver

\subsubsection{Antecedentes y Causas del Problema}
    - Detalle de los antecedentes y causas del problema
    - Indicación de la importancia de resolverlo

% =============================================
\subsection{Generación y Justificación}
% =============================================

\subsubsection{Metodología y Herramientas Propuestas}
    - Propone una metodología, herramientas y/o modelos para el análisis, diseño, desarrollo y solución del objetivo/problema (utiliza supuestos si no fueron aplicados durante el periodo de práctica).
    - Utiliza recursos visuales (e.g: como diagramas de flujo, figuras, gráficos, tablas, etc.) que facilitan el entendimiento de las labores realizadas o metodologías utilizadas.

\subsubsection{Análisis, Herramientas y Software para la Resolución del Problema}
    - Describe los análisis, mediciones o aplicaciones y los softwars necesarios para la resolución del problema/demanda (utiliza supuestos si no fueron aplicados durante el periodo de práctica).
    - Si no se hizo antes, utilizar recursos visuales


\subsubsection{Justificación de decisiones metodológicas y de recursos}
    - Justifica las decisiones tomadas en cuanto a la metodología o los recursos utilizados de acuerdo a las características del problema/objetivo abordado y del lugar de realización de la práctica.
    - Si no se hizo antes, utilizar recursos visuales

% =============================================
\subsection{Evaluación y Conclusiones}
% =============================================

\subsubsection{Resultados y Análisis}
    - Describe los resultados obtenidos utilizando indicadores concretos o mediciones reales o estimadas.

\subsubsection{Conclusiones sobre los Resultados}
    - Concluye acerca de los resultados obtenidos, considerando limitaciones y relevancia de los análisis realizados.
    - Discute las implicancias futuras de las decisiones tomadas y plantea posibles modificaciones y pasos a seguir.

\subsubsection{Evaluación de Cumplimientos de Objetivos de la Empresa}
    - Concluye acerca del cumplimiento o incumplimiento de los objetivos o intereses de la empresa.

\subsubsection{Impacto y Relevancia de los Resultados}
    - Justifica adecuadamente el impacto o relevancia de los resultados obtenidos, considerando aspectos como eficiencia, productividad, clima laboral, etc.

