\pregunta[5] % pregunta[puntaje]
Dada la curva $2y^3+2x^3+y^2-y^5=x^4+x^2$, encuentre \textbf{todos} los valores de x donde la primera derivada es cero.
\mostrarpuntaje

\begin{alternativas}
   \alternativa  $x = \{-1, 0\}$
   \alternativaCorrecta$x = \{\frac{1}{2}, 1,0\}$
   \alternativa $x = \{1, 0\}$
   \alternativa Ninguna de las anteriores.
\end{alternativas}
%
\pregunta[5]
Encuentre la ecuación de la recta tangente a la siguiente función en el punto pedido:
$$x^2 + 4xy + y^2 = 13 \text{, en el punto }(2,1)$$
\mostrarpuntaje
\begin{alternativas}
   \alternativaCorrecta $f(x) = \frac{-4}{5}x + \frac{13}{5}$
   \alternativa $f(x) = \frac{5}{4}x + \frac{9}{4}$
   \alternativa $f(x) = -\frac{5}{4}x + \frac{9}{4}$
   \alternativa $f(x) = \frac{2}{5}x - \frac{1}{5}$
\end{alternativas}
%
\pregunta[5]
Dos autos ({\emoji \symbol{"1F697}}) A y B viajan por dos calles perpendiculares desde la intersección de  estas. El auto A viaja a $\SI{30}{\kilo\meter\per\hour}$ y el auto B a $\SI{70}{\kilo\meter\per\hour}$ ¿A qué velocidad se alejan los autos cuando A se encuentra a $\SI{3}{\kilo\meter}$ de la intersección?
\mostrarpuntaje

\begin{alternativas}
   \alternativa  $\frac{\sqrt{58}}{1160}$
   \alternativaCorrecta$\frac{580}{\sqrt{58}}$
   \alternativa $\frac{\sqrt{58}}{580}$
   \alternativa $\frac{\sqrt{1160}}{58}$
\end{alternativas}
%
\pregunta[5]
Compute el siguiente límite:
$$\lim_{x\to0} \frac{e^x - e^{-x}}{\sin{x}}$$
\mostrarpuntaje

\begin{alternativas}
   \alternativa $0$
   \alternativa $1$
   \alternativaCorrecta$2$
   \alternativa $\infty$
\end{alternativas}
%
\pregunta[5]
Se instala una cámara de televisión a $\SI{12000}{\meter}$ de la base de una plataforma de lanzamiento de cohetes. ({\emoji \symbol{"1F680}}) El  ángulo de elevación de la cámara tiene que cambiar con la proporción correcta con el objeto de tener siempre a la vista al cohete. Asimismo, el mecanismo de enfoque de la cámara tiene que tomar en cuenta la distancia creciente de la cámara al cohete que se eleva. Suponga que el cohete se eleva verticalmente y que su rapidez es $\SI{800}{\meter\per\second}$ cuando se ha elevado $\SI{5000}{\meter}$. ¿Qué tan rápido cambia el ángulo de elevación de la cámara en ese momento? (Pista: usa tangente)
\mostrarpuntaje

\begin{alternativas}
   \alternativaCorrecta$\frac{48}{845}$
   \alternativa $\frac{144}{169}$
   \alternativa $\frac{4000}{13}$
   \alternativa $\frac{5}{17}$
\end{alternativas}
%
\pregunta[5]
Halle la linealización de $f(x) = \sqrt[\leftroot{2}\uproot{2}3]{1+3x}$ en $x=0$.
\mostrarpuntaje

\begin{alternativas}
   \alternativaCorrecta$L(x) = x + 1$
   \alternativa $L(x) = -x + 1$
   \alternativa $L(x) = 3x + 1$
   \alternativa $L(x) = x^3 + 1$
\end{alternativas}
%
\pregunta[5]
Compute el siguiente límite:
$$\lim_{x\to0}\left(\frac{1}{\ln{(1+x)}}-\frac{1}{x}\right)$$
\mostrarpuntaje

\begin{alternativas}
   \alternativaCorrecta$\frac{1}{2}$
   \alternativa $0$
   \alternativa $-\infty$
   \alternativa $2$
\end{alternativas}
