% -----------------------------------------------------
% QUESTION
% -----------------------------------------------------
\question 
Mediante el teorema del valor medio (TVM), demuestre que si $x > 0$, entonces $\arctan(x) < x$.
\begin{solution}
Debemos comenzar definiendo el rango de valores de $x$ vamos a trabajar. Sabemos que $\arctan$ es una función continua en todo los R, al igual que $x$. Sin embargo, tenemos la condición de $x > 0$, por lo que nuestro rango de valores se acota a $(0, x)$, donde x puede tomar cualquier valor. \\

Se define una función auxiliar llamada $h(x)$ la cual será igual a el valor mas grande de la desigualdad, menos el valor mas chico de la desigualdad. Dado eso, sabemos que $h(x)$ siempre será positivo. \\

Luego, establecemos la condición del TVM, $\frac{f(b) - f(a)}{b - a } = f'(c)$. Reemplazando con nuestra función sabemos que $b = x$ y $a = 0$ (por el intervalo dado).

Ahora reemplazamos, $\frac{h(x) - h(0)}{x-0} = h'(c)$. (Evaluando 0 en $h(x)$ nos queda $0 - \arctan(0) = 0$). Despejando $h(x)$ de lo anterior:
$$h(x) = h'(c) \cdot x$$
Notamos que $x > 0$. Por lo que necesitamos saber si $h'(c) > 0$.
Derivando $h(x)$ para encontrar su signo, y evaluando en $c$ tenemos: 
$$h'(c) = 1 - \frac{1}{1 + c^2}$$Por hipótesis del teorema del valor medio, $0 < c < x$, por lo que nuestra fracción será $>0$ y $<1$, dando como resultado que $h'(c) > 0$.  Ahora que sabemos que $x>0$, $h'(c) > 0$ y $h(x) > 0$, reemplazamos en la expresión $h(x) = h'(c) \cdot x > 0$, quedando: $$(x - \arctan(x) = (1 - \frac{1}{1 + c^2})\cdot x > 0$$Y ahora por transitividad de la desigualdad $x - \arctan x > 0$ y despejando $x$, $x > \arctan x$.
\end{solution}