% ----------------------------------------------------------
\question[5]
Determina la recta tangente de la curva $x^2 + y^2 = (2x^2 + 2y^2 - x)^2$  para el punto $(0, \frac{1}{2})$.
\droppoints

\begin{checkboxes}
\choice  $f(x) = -x + \frac{1}{2}$
\CorrectChoice $f(x) = x + \frac{1}{2}$
\choice $f(x) = -x - \frac{1}{2}$
\choice $f(x) = x - \frac{1}{2}$
\end{checkboxes}

% ----------------------------------------------------------
\question[5]
Dada la curva $2y^3+2x^3+y^2-y^5=x^4+x^2$, encuentre \textbf{todos} los valores de x donde la primera derivada es cero.
\droppoints

\begin{checkboxes}
\choice  $x = \{-1, 0\}$
\CorrectChoice $x = \{\frac{1}{2}, 1,0\}$
\choice $x = \{1, 0\}$
\choice Ninguna de las anteriores.
\end{checkboxes}
% ----------------------------------------------------------
\question[5]
Encuentre la ecuación de la recta tangente a la siguiente función en el punto pedido:
$$x^2 + 4xy + y^2 = 13 \text{, en el punto }(2,1)$$
\droppoints
\begin{checkboxes}
\CorrectChoice  $f(x) = \frac{-4}{5}x + \frac{13}{5}$
\choice $f(x) = \frac{5}{4}x + \frac{9}{4}$
\choice $f(x) = -\frac{5}{4}x + \frac{9}{4}$
\choice $f(x) = \frac{2}{5}x - \frac{1}{5}$
\end{checkboxes}
% ----------------------------------------------------------
\question[5]
Encuentre la aproximación lineal de $f(x)=\sqrt{-x^2 + 25}$ en el punto $x=3$.
\droppoints

\begin{checkboxes}
\choice  $L(x) = \frac{3}{4}x + \frac{7}{4}$
\choice $L(x) = \frac{-3}{4}x - \frac{25}{4}$
\CorrectChoice $L(x) = \frac{-3}{4}x + \frac{25}{4}$
\choice $L(x) = \frac{3}{4}x - \frac{7}{4}$
\end{checkboxes}
% ----------------------------------------------------------
\question[5]
Dada la función $f(x)=x^3-3x+2$ identifique los valores máximo y mínimo absolutos para el rango $[-2, 2]$. ¿Cuál de las siguientes afirmaciones es verdadera?
\droppoints

\begin{checkboxes}
\choice Existe un máximo absoluto en $x = -2$, con 4 de altura.
\choice Existe un mínimo absoluto en $x = -1$, y yace en el eje x.
\choice Existen máximos absoluto en $x = -2$ y $x = 1$ y ambos yacen en el eje x.
\CorrectChoice Existe un máximo absoluto en $x = 2$ y $x = -1$, con valor 4.
\end{checkboxes}
% ----------------------------------------------------------
\pagebreak
\question[5]
Si $a$ y $b$ son números reales positivos, el valor máximo absoluto de $f(x) = x^a (1-x)^b$ en el intervalo $[0,1]$ es:
\droppoints

\begin{checkboxes}
\CorrectChoice $\frac{a^a \cdot b^b}{(a + b)^{a + b}}$
\choice $\frac{a^a \cdot b^b}{(a - b)^{a - b}}$
\choice $-\frac{a^a \cdot b^b}{(a + b)^{a + b}}$
\choice $\frac{a^a \cdot b^b}{(a - b)^{a + b}}$
\end{checkboxes}
% ----------------------------------------------------------
\question[5]
Dos autos ({\emoji \symbol{"1F697}}) A y B viajan por dos calles perpendiculares desde la intersección de  estas. El auto A viaja a $\SI{30}{\kilo\meter\per\hour}$ y el auto B a $\SI{70}{\kilo\meter\per\hour}$ ¿A qué velocidad se alejan los autos cuando A se encuentra a $\SI{3}{\kilo\meter}$ de la intersección?
\droppoints

\begin{checkboxes}
\choice  $\frac{\sqrt{58}}{1160}$
\CorrectChoice $\frac{580}{\sqrt{58}}$
\choice $\frac{\sqrt{58}}{580}$
\choice $\frac{\sqrt{1160}}{58}$
\end{checkboxes}
% ----------------------------------------------------------
\question[5]
Compute el siguiente límite:
$$\lim_{x\to0} \frac{e^x - e^{-x}}{\sin{x}}$$
\droppoints

\begin{checkboxes}
\choice $0$
\choice $1$
\CorrectChoice $2$
\choice $\infty$
\end{checkboxes}
% ----------------------------------------------------------
\question[5]
Se instala una cámara de televisión a $\SI{12000}{\meter}$ de la base de una plataforma de lanzamiento de cohetes. ({\emoji \symbol{"1F680}}) El  ángulo de elevación de la cámara tiene que cambiar con la proporción correcta con el objeto de tener siempre a la vista al cohete. Asimismo, el mecanismo de enfoque de la cámara tiene que tomar en cuenta la distancia creciente de la cámara al cohete que se eleva. Suponga que el cohete se eleva verticalmente y que su rapidez es $\SI{800}{\meter\per\second}$ cuando se ha elevado $\SI{5000}{\meter}$. ¿Qué tan rápido cambia el ángulo de elevación de la cámara en ese momento? (Pista: usa tangente)
\droppoints

\begin{checkboxes}
\CorrectChoice $\frac{48}{845}$
\choice $\frac{144}{169}$
\choice $\frac{4000}{13}$
\choice $\frac{5}{17}$
\end{checkboxes}
% ----------------------------------------------------------
\pagebreak
\question[5]
Halle la linealización de $f(x) = \sqrt[\leftroot{2}\uproot{2}3]{1+3x}$ en $x=0$.
\droppoints

\begin{checkboxes}
\CorrectChoice $L(x) = x + 1$
\choice $L(x) = -x + 1$
\choice $L(x) = 3x + 1$
\choice $L(x) = x^3 + 1$
\end{checkboxes}
% ----------------------------------------------------------
\question[5]
Compute el siguiente límite:
$$\lim_{x\to0}\left(\frac{1}{\ln{(1+x)}}-\frac{1}{x}\right)$$
\droppoints

\begin{checkboxes}
\CorrectChoice $\frac{1}{2}$
\choice $0$
\choice $-\infty$
\choice Solo Thor-res sabe.
\end{checkboxes}
